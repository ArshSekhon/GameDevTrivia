\subsection*{Introduction}

Game Development Trivia is a Trivia game based upon Allegro 4.\+4.\+2. The project was created using Visual Studio.

The game basically reads the questions and answers provided in the text files in the questions directory and uses them to create quiz consisting of 10 questions for the Players. These questions are derived from the back exercises of the book Game Programming All in One by Jonathan S. Harbour.

Player can choose the chapters he/she wishes the quiz to be on. Key features of the Game include\+:
\begin{DoxyItemize}
\item Beatiful graphics
\item Amazing sound effects and music
\item Sound and Graphics related settings that are persistent
\item User friendly user interfaces
\item Game supports 3 different resolutions, full screen and windowed mode
\end{DoxyItemize}

{\itshape Note\+: Some graphics settings like high resolution or full screen mode may misbehave on newer version of windows due to compatibility issue as Allegro 4.\+4.\+2 came out in early 2000s}

\subsection*{Install Instructions}

In order to run the game you would need to setup the allegro first\+:


\begin{DoxyEnumerate}
\item Go to \href{https://www.allegro.cc/files/?v=4.4}{\texttt{ https\+://www.\+allegro.\+cc/files/?v=4.\+4}} and download windows binaries for M\+S\+VC 10
\item Unzip the binaries to C\+:/allegro
\item Now you should have the following folders C\+:/allegro/bin, C\+:/allegro/include and C\+:/allegro/lib.
\item On the Windows desktop, right-\/click My Computer and click properties.
\item In the System Properties window, go to the following Advanced tab $>$ Environment Variables $>$ System Variables.
\item Find the variable Path and click edit, if the last character is not a semi-\/colon(;) then add one ; to the end and after it paste C\+:/allegro/bin;
\item Click apply a bunch of times as required and you should be good to go.
\end{DoxyEnumerate}

Now open the Pocket\+Trivia Project in Visual Studio and hit run!

\subsection*{Technical Design}

The entrypoint for the project is Pocket\+Trivia.\+cpp. The file basically creates a \mbox{\hyperlink{class_game_manager}{Game\+Manager}} object and calls the \mbox{\hyperlink{class_game_manager_a0b2f746268718b7fccc05c178217fcff}{game\+Manager.init()}} function. This function does all the required setup and perform tasks like setting up allegro, creating struct to store game\+State, acquiring mouse, setting up sound etc.

After this \mbox{\hyperlink{class_game_manager_a7a840d42c2e52b2f827800bbac51c944}{game\+Manager.run\+Game\+Loop()}} is called and the control is completely passed to the game manager. The game loop basically checks the current state of the game and then hands over the control to object of correct Screen/\+Menu class. These Screen/\+Menu objects basically take care of things like drawing graphics on screen, playing sounds, handling event etc.

For e.\+g. if the Game state indicates that player has settings menu open then game loop would basically call \mbox{\hyperlink{class_settings_menu_a5df9575a1e6e76b4159b417e4f66b222}{settings\+Menu.show\+Settings\+Menu}} This would draw the required graphics on the screen, would do the required event handling and actions by making use of other classes

This is a brief overview of how the game is designed.

Refer to Game\+Manager.\+cpp for more details. 